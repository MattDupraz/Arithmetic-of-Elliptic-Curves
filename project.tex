\documentclass{article}

\usepackage{amsmath, amsfonts, amssymb}
\usepackage{amsthm}
\usepackage{hyperref}

\author{Matthew Dupraz}
\title{Elliptic curves over $\mathbb{C}$ and over finite fields}

\newtheorem{theorem}{Theorem}[section]
\newtheorem{corollary}{Corollary}[theorem]
\newtheorem{lemma}[theorem]{Lemma}
\newtheorem{proposition}[theorem]{Proposition}

\theoremstyle{definition}
\newtheorem{definition}{Definition}[section]
\newtheorem*{notation}{Notation}

\theoremstyle{remark}
\newtheorem*{remark}{Remark}

\newcommand{\proj}{\mathbb{P}}
\newcommand{\F}{\mathbb{F}}
\newcommand{\C}{\mathbb{C}}
\newcommand{\chr}{\mathrm{char}}

\begin{document}

\maketitle

\subsection{Weierstrass equation}

Our main interest are \emph{elliptic curves}, which are curves
in $\proj^2$ of genus 1. These are characterized by the
homogeneous equation
\begin{equation}
	\label{weierstrass-eq-1}
	Y^2Z + aXYZ + bYZ^2 = X^3 + cX^2Z + dXZ^2 + eZ^3
\end{equation}
for some $a, b, c, d, e \in \F$.
Setting $U_Z = \{[X, Y, Z] \in \proj^2 \mid Z \neq 0\}$, we can
study the solutions of (\ref{weierstrass-eq-1}) on
$U_Z$ using the change of coordinates $x = X/Z$
and $y = Y/Z$. We obtain the following equation
\begin{equation}
	\label{weierstrass-eq-2}
	y^2 + axy + by = x^3 + cx^2 + dx + e
\end{equation}
We can further simplify this equation with linear
changes of variables. First notice that if $\chr(\F) \neq 2$,
the left hand side can be
written as
\begin{align*}
	y(y + ax + b) &= (y + \frac{1}{2}(ax + b) - \frac{1}{2}(ax + b))
	(y + \frac{1}{2}(ax + b) + \frac{1}{2}(ax + b))\\
	&= (y + \frac{1}{2}(ax + b))^2 - \frac{1}{4}(ax + b)^2
\end{align*}
Hence by replacing $y$ with $y + \frac{1}{2}(ax + b)$ and
collecting the terms in each monomial, we get an equation
of the form
\begin{equation}
	y^2 = x^3 + \alpha x^2 + \beta x + \gamma
\end{equation}
If $\chr(\F) \neq 3$, we can also get rid of the term in 
$x^2$ with a linear change
of variables. replacing $x$ with $x - \frac{1}{3}\alpha$ yields
\begin{align*}
	y^2 &= (x - \frac{1}{3}\alpha)^3 + \alpha(x-\frac{1}{3}\alpha)^2
	+ \beta(x - \frac{1}{3}\alpha) + \gamma\\
	&= x^3 - \alpha x^2 + \frac{1}{3}\alpha^2 x
	- \frac{1}{27}\alpha^3 + \alpha x^2 - \frac{2}{3}\alpha^2 x
	+ \frac{1}{9}\alpha^3 + \beta x - \frac{1}{3}\alpha \beta
	+ \gamma
\end{align*}
Collecting the terms in each monomial, we get an equation of the
form
\begin{equation}
	y^2 = x^3 + Ax + B
\end{equation}
with $A, B \in \F$.
Plugging back the substitutions $x = X/Z$ and $y = Y/Z$, we obtain
the homogeneous equation
\begin{equation}
	Y^2Z = X^3 + AXZ^2 + BZ^3
\end{equation}

\subsection{Singularities}

We suppose $\F$ is algebraically closed.
	
We have that an elliptic curve $V \subset \proj_2(\F)$
is the projective variety
\begin{equation}
	V = V(X^3 + AXZ^2 + BZ^3 - Y^2Z) = V(F)
\end{equation}
We are interested in the case where the curve is smooth.
By the regular preimage theorem, $V$ is smooth if all its points
are non-singular, i.e. if for all $P = [x, y, z] \in V$,
\begin{equation*}
	\nabla F(P) = 
	\begin{bmatrix}
		3x^2 + Az^2\\
		-2yz\\
		2Axz + 3Bz^2 - y^2
	\end{bmatrix}
	\neq 0
\end{equation*}
If $P = [0, 1, 0]$, then 
\begin{equation*}
	\nabla F(P) = 
	\begin{bmatrix}
		0\\
		0\\
		-1
	\end{bmatrix} \neq 0
\end{equation*}
hence the point at infinity is never singular. It follows
that when looking for singularities, we can consider just
the case where $z \neq 0$, since else we have necessarily $x = 0$
and so $P = [0, 1, 0]$. So if there are any singularities of $V$,
they are on $V \cap U_Z$. So $V$ is non-singular precisely when
$V \cap U_Z$ is non-singular. Using the isomorphism
$V \cap U_Z \to W, [X, Y, Z] \mapsto (\frac{X}{Z}, \frac{Y}{Z})$ it
suffices to study singularities on $W = V(x^3 + Ax + B - y^2)
= V(f)$

Let $\Delta = 4A^3 + 27B^2$ be the discriminant of the polynomial
$g(x) = x^3 + Ax + B$, we have the following criteria for the
existence of singularities of $V$.

\begin{proposition}
	W (and equivalently V)
	is non-singular if and only if $\Delta \neq 0$.
\end{proposition}
\begin{proof}
	Suppose there is a point $P = (x_0, y_0) \in W$ that is singular,
	then we have
	\begin{equation*}
		\begin{bmatrix}
			3x_0^2 + A\\
			-2y_0\\
		\end{bmatrix} = 0
	\end{equation*}
	Hence we have that $g'(x_0) = 3x_0^2 + A = 0$ and $y_0 = 0$.
	In particular, since $P \in W$, also 
	$g(x_0) = 0$, and hence since $g(x_0) = g'(x_0) = 0$,
	$x_0$ is a double root of $g$ and so the discriminant
	$\Delta = 4A^3 + 27B^2$ of $g$ is zero.

	Suppose instead that $\Delta = 0$, then $g$ admits a double root
	$x_0 \in \F$ (since we supposed $\F$ algebraically closed)
	which is unique since $g$ is a cubic polynomial.
	Then $P = (x_0, 0) \in V$.
	Furthermore,
	\begin{equation*}
		\nabla f(P) =
		\begin{bmatrix}
			3x^2 + A\\
			0\\
		\end{bmatrix}
	\end{equation*}
	We have that $3x^2 + A = g'(x) = 0$, hence $\nabla f(P) = 0$
	and so $W$ is singular at $P$.
\end{proof}

\subsection{Elliptic curves over \texorpdfstring{$\C$}{C}}

The goal of this section is to show an elliptic curve is
homeomorphic to a torus.

\begin{definition}
	Let $\Lambda \subseteq \C$ be a lattice
	\begin{enumerate}
		\item
			The Weierstrass
			elliptic function ($\wp$-function),
			is defined by the series
			\begin{equation*}
				\wp(z; \Lambda) = \frac{1}{z^2}
				+ \sum_{\lambda \in \Lambda\setminus\{0\}}
				\left(
					\frac{1}{(z-\lambda)^2} - \frac{1}{\lambda^2}
				\right)
			\end{equation*}
		\item
			The Eisenstein series (of $\Lambda$) of weight $k$,
			where $k \geq 2$ is an integer
			is the series
			\begin{equation*}
				G_k(\Lambda) = \sum_{\lambda \in \Lambda\setminus\{0\}}
				\lambda^{-k}
			\end{equation*}
	\end{enumerate}
\end{definition}

\begin{notation}
	If $\Lambda$ is known from context, we write simply
	$\wp(z)$ and $G_k$ for $\wp(z; \Lambda), G_k(\Lambda)$
	respectively.
\end{notation}

% Theorem 3.1
\begin{proposition}
	Let $\Lambda$ be a lattice.
	\begin{enumerate}
		\item	The Eisenstein series $G_k(\Lambda)$ is absolutely convergent
			for all $k \geq 3$.
		\item The series defining the Weierstrass $\wp$-function converges
			absolutely and uniformly on every compact subset of
			$\C\setminus\Lambda$. It defines a meromorphic function on $\C$ with 
			double poles of residue 0 at each lattice point.
		\item The Weierstrass $\wp$-function is an even elliptic function.
	\end{enumerate}
\end{proposition}

\begin{proof}
	\begin{enumerate}
	\item	Let $\lambda_1, \lambda_2$ be basis vectors of $\Lambda$
		of minimal length, with $|\lambda_1| \leq |\lambda_2|$.
		Let $A_N := \{n\lambda_1 + m\lambda_2 \in \Lambda \mid
		\max(|n|, |m|) = N\}$.
		Then 
		\begin{equation*}
			\#A_N = (2N + 1)^2 - (2N - 1)^2 = 8N.
		\end{equation*}
		Furthermore, $\min\{|\lambda|, \lambda \in A_N\} = N|\lambda_1|$, so we
		get
		\begin{equation*}
			\sum_{\lambda \in \Lambda\setminus 0}\frac{1}{|\lambda|^k}
			\leq \sum_{N=1}^\infty \frac{\#A_N}{\min\{|\lambda|, \lambda \in
			A_N\}^k}
			= \sum_{N=1}^{\infty} \frac{8}{|\lambda_1|N^{k-1}} < \infty.
		\end{equation*}

	\end{enumerate}
\end{proof}

% Proposition VI.3.6
\begin{theorem}
	Let $\Lambda \subseteq \C$ be a lattice
	and $g_2, g_3$ its associated quantities.
	Let $E/\C$ be the curve given by the equation
	\begin{equation*}
		E: y^2 = 4x^3 - g_2 x - g_3
	\end{equation*}
	then $E$ is an elliptic curve and the map
	\begin{align*}
		\phi: \C/\Lambda &\to E\\
		z &\mapsto [\wp(z), \wp'(z), 1]
	\end{align*}
	is a complex analytic isomorphism of complex Lie groups.
\end{theorem}

% Theorem VI.5.1
\begin{theorem}
	Let $A, B \in \C$ that satisfy $A^3 - 27B^2 \neq 0$.
	Then there exists a unique lattice
	$\Lambda \subseteq \C$ such that
	$g_2(\Lambda) = A$ and $g_3(\Lambda) = B$.
\end{theorem}

\end{document}
