\section*{Introduction}

In this paper, we will follow closely the structure of
Silverman's \emph{Arithmetic of Elliptic
Curves} \cite{silverman} with the goal to state the Weil conjectures,
and prove them for the case of elliptic curves, which are projective curves
of genus 1.
The Weil conjectures give some properties of the zeta function,
which is a generating
function derived from counting points on algebraic varieties over finite
fields.
In particular, one of the conjectures gives a link between
the rank of homology groups of algebraic varieties defined over the complex 
numbers and the form of the zeta function of a related algebraic variety
defined over a finite field. To make this link for the case of
elliptic curves, in Section \ref{sec:finite-fields}, we will for one calculate
the zeta function for elliptic curves defined over finite fields, and then
in Section \ref{sec:over-C}, we will study the topological structure of
elliptic curves defined
over complex numbers.

We will start by defining all the necessary notions to be able to understand
the Weil conjectures and define important constructions that will be used
in the proof of the Weil conjectures. 
Throughout this paper we assume known the content of the course \emph{Algebraic
Curves} given by Dimitri Wyss.

Whenever we talk about algebraic varieties defined over a field $K$, we
will assume $K$ is an algebraically closed field, unless stated otherwise.
Furthermore, throughout this paper, we will assume that
$\chr(K) \not\in \{2, 3\}$, as these cases lead to subtleties that go outside
the scope of this paper.
