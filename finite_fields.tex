\section{Weil Conjectures}
\label{sec:finite-fields}

For this section we fix a prime $p$ and $q$ a power of $p$.
We suppose throughout this section that $\chr(K) = p$.

In this section we will state the Weil conjectures and prove them in the
case of elliptic curves.

If $K$ is of characteristic $p$,
it contains a unique subfield of order $p^n$ for any $n \in \N$ (see
course \emph{Rings and Fields}), we will denote this subfield by $\F_{p^n}$.
We will be studying the set of
$\F_{q^n}$-rational points of a projective variety.

\begin{definition}
	Let $V/\F_q$ be a projective variety.
	The zeta function of $V/\F_q$ is defined as the power series
	\begin{equation*}
		Z(V/\F_q; T) = \exp\left(\sum_{n=1}^\infty (\#V(\F_{q^n}))
		\frac{T^n}{n}\right)
	\end{equation*}
\end{definition}

\begin{notation}
	When $V/\F_q$ is known from context, we write simply $Z(T)$
	instead of $Z(V/\F_q; T)$
\end{notation}

\begin{theorem}[Weil Conjectures]
	\label{thm:weil}
	Let $V/\F_q$ be a smooth projective variety of dimension $N$.
	\begin{enumerate}[label=(\alph*)]
		\item Rationality: $Z(T) \in \Q(T)$. More precisely, 
			there is a factorization
			\begin{equation*}
				Z(T) = \frac{P_1(T)\cdots P_{2n-1}(T)}
				{P_0(T)P_2(T) \cdots P_{2n}(T)},
			\end{equation*}
			where $P_0(T) = 1 - T, P_{2n}(T) = 1 - q^nT$ and for each
			$1 \leq i \leq 2n - 1$, $P_i(T)$ factors (over $\C$) as
			\begin{equation*}
				P_i(T) = \prod_j (1 - \alpha_{ij}T)
			\end{equation*}
		\item Functional Equation: The zeta function satisfies
			\begin{equation*}
				Z\left(\frac{1}{q^NT}\right) = \pm q^{N\frac{\epsilon}{2}}
				T^{\epsilon} Z(T),
			\end{equation*}
			for some integer $\epsilon$ (called the Euler characteristic of $V$)
		\item Riemann Hypothesis: $|\alpha_{ij}| = q^{i/2}$
			for all $1 \leq i \leq 2n - 1$ and all $j$.
		\item Betti Numbers: If $V/\F_q$ is a good reduction mod $p$ of a
			non-singular projective variety $W/K$, where $K$ is a number
			field embedded in the field of complex numbers, then the degree
			of $P_i$ is the $i$\textsuperscript{th} Betti number (i.e.
			rank of the $i$\ts{th} homology group) of the space
			of complex points of $W$. 
	\end{enumerate}
\end{theorem}

We won't define what a ``good reduction" means in general, but we can
look at the case of elliptic curves given by a Weierstrass equation.

If $K$ be a number field (seen as a subfield of its algebraic closure $\cl{K}$)
and $\O$ its ring of integers. Suppose $E$ is an elliptic curve given by a
Weierstrass equation defined over $K$, i.e. $E$ is of the form
\begin{equation*}
	E: y^2 = x^3 + ax + b
\end{equation*}
with $a, b \in K$. We have that $K = \Frac(\O)$ and so we can write
$a = a_1/a_2$ and $b = b_1/b_2$ for some $a_1, b_1 \in \O$, and
$a_2, b_2 \in \O\setminus\{0\}$.
We can (uniquely) decompose the ideal $(a_2b_2)$ into a product of prime ideals
(see course \emph{Algebraic Number Theory}).
\begin{equation*}
	(a_2b_2) = \mf{p}_1\dots\mf{p}_s
\end{equation*}
Then choosing a prime ideal
$\mf{p}$ of $\O$ different from $\mf{p}_i$ for any $i$,
we get that $a_2, b_2 \notin \mf{p}$.
Hence we can see $E$ as being defined over $\O_\mf{p}$.
The maximal ideal $\mf{P}$ of $\O_\mf{p}$ is just the image of $\mf{p}$
under the localization. We deduce
\begin{equation*}
	\O_\mf{p}/\mf{P} \cong (\O/\mf{p})_\mf{p} = \O/\mf{p} \cong \F_q,
\end{equation*}
where $q$ is a power of $p$, where $(p) = \mf{p}\cap\Z$.

This gives us a new curve $C$ obtained by reducing $E$ modulo $\mf{P}$.

\begin{equation*}
	C: y^2 = x^3 + \bar{a}x + \bar{b},
\end{equation*}
defined over the residue field isomorphic to $\F_q$.
We say that $C$ is a \emph{good} reduction of $E$ modulo $p$
if it is also smooth. That is the case if and only if its discriminant
\begin{equation*}
	\Delta(C) = 4\bar{a}^3 + 27\bar{b}^2
\end{equation*}
is non-zero. But notice that the discriminant of $C$ is just the residue
of $\Delta(E) = 4a^3 + 27b^2$ modulo $\mf{P}$.
Hence the reduction of $E$ mod $p$ is ``good" if $\Delta(E) \not\in \mf{P}$.
We will show that (d) of \ref{thm:weil} holds for elliptic curves given
by Weierstrass equations in Section
\ref{sec:over-C}.

In the rest of this section, we will prove the Weil conjectures (save part (d))
for the case
of elliptic curves. For that, we will make use of the relation found in
\ref{prop:deg-tr-det}.
In the following proposition we get a formula for $\#E(\F_{q^n})$, which
we will be able to use for proving the Weil conjectures.

\begin{proposition}
	\label{prop:frob-char-poly}
	Let $E/\F_q$ be an elliptic curve, and
	\begin{equation*}
		\phi: E \to E, (x, y) \mapsto (x^q, y^q)
	\end{equation*}
	the $q$\ts{th}-power Frobenius endomorphism.
	Let $\alpha, \beta \in \C$ be the roots of the characteristic polynomial
	of $\phi_l$, that is
	\begin{equation*}
		\det(T - \phi_l) = T^2 - \tr(\phi_l)T + \det(\phi_l),
	\end{equation*}
	then $\alpha, \beta$ are complex conjugates satisfying
	$|\alpha| = |\beta| = \sqrt{q}$. Furthermore, for every $n \geq 1$, we
	have
	\begin{equation*}
		\#E(\F_{q^n}) = q^n + 1 - \alpha^n - \beta^n
	\end{equation*}
\end{proposition}

\begin{proof}
	We have by \ref{prop:frobenius-separable}
	and \ref{thm:preimage-card}
	that
	\begin{equation*}
		\#E(\F_q) = \deg(1 - \phi)
	\end{equation*}
	and from \ref{prop:deg-tr-det}, we have that
	\begin{equation*}
		\det(\phi_l) = \deg(\phi) = q;\\
	\end{equation*}
	For all $m/n \in \Q$, with $p \nmid m$,
	we have using \ref{prop:frobenius-separable} that
	\begin{equation*}
		\det\left(\frac{m}{n} - \phi_l\right)
		= \frac{\det(m - n\phi_l)}{n^2}
		= \frac{\deg(m - n\phi_l)}{n^2} \geq 0
	\end{equation*}
	Hence the polynomial $\det(T - \phi_l)$ is non-negative for $T \in \R$
	(by continuity). If $\alpha$, $\beta$ are the roots of $\det(T - \phi_l)$,
	it follows that $\alpha$, $\beta$ are complex conjugates (they can be
	equal). So $|\alpha| = |\beta|$ and since $\alpha\beta = \det(\phi_l) = q$,
	it follows that $|\alpha| = |\beta| = \sqrt{q}$.

	Now, for $n \geq 1$ the $(q^n)$\ts{th}-power Frobenius endomorphism
	$\phi^n$ satisfies
	\begin{equation*}
		\#E(\F_{q^n}) = \deg(1 - \phi^n) = \det(1 - \phi_l^n)
	\end{equation*}
	We have that
	\begin{equation*}
		\det(T - \phi_l^n) = (T - \alpha^n)(T - \beta^n)
	\end{equation*}
	since the eigenvalues of $\phi_l^n$ are the $n$\ts{th} powers
	of the eigenvalues of $\phi_l$. From 
	\ref{prop:deg-tr-det}, we have that
	\begin{equation*}
		\tr(\phi_l^n) = 1 + \deg(\phi^n) - \deg(1 - \phi^n)
		= 1 + q^n - \#E(\F_{q^n}).
	\end{equation*}
	and hence 
	\begin{equation*}
		\#E(\F_{q^n}) = 1 + q^n - \tr(\phi_l^m)
		= 1 + q^n - \alpha^n - \beta^n.
	\end{equation*}
\end{proof}

At last, we can state and prove the Weil conjectures for the case
of elliptic curves.

\begin{theorem}
	\label{thm:weil-elliptic}
	Let $E/\F_q$ be an elliptic curve. Then there exists an $a \in \Z$ such that
	\begin{equation*}
		Z(T) = \frac{1 - aT + qT^2}{(1-T)(1-qT)}.
	\end{equation*}
	Furthermore,
	\begin{equation*}
		Z\left(\frac{1}{qT}\right) = Z(T)
	\end{equation*}
	and
	\begin{equation*}
		1 - aT + qT^2 = (1 - \alpha T)(1 - \beta T)
	\end{equation*}
	with $|\alpha| = |\beta| = \sqrt{q}$
\end{theorem}

\begin{proof}
	Using the definition of $Z(E/\F_q; T)$, we get
	\begin{align*}
		\log Z(E/\F_q; T) &= \sum_{n=1}^\infty (\#E(\F_{q^n}))\frac{T^n}{n}\\
		&= \sum_{n=1}^\infty (q^n + 1 - \alpha^n - \beta^n)\frac{T^n}{n}
		\qquad \textrm{(\ref{prop:frob-char-poly})}\\
		&= -\log(1 - qT) - \log(1 - T) + \log(1 - \alpha T) + \log(1 - \beta T)
	\end{align*}
	and hence we get
	\begin{equation*}
		Z(E/\F_q; T) = \frac{(1 - \alpha T)(1 - \beta T)}{(1 - T)(1 - qT)},
	\end{equation*}
	which has the desired form.
	Indeed from (\ref{prop:frob-char-poly}), 
	$|\alpha| = |\beta| = \sqrt{q}$, and
	\begin{align*}
		a = \alpha + \beta = \tr(\phi_l) &= 1 + \deg(\phi) - \deg(1 - \phi)\\
		&= 1 + q - \#E(\F_q) \in \Z.
	\end{align*}
\end{proof}

Hence the Weil conjectures are verified for elliptic curves. Notice that
using the notation from theorem \ref{thm:weil},
$\deg P_0 = 1$, $\deg P_1 = 2$, $\deg P_2 = 1$, hence if $C/\F_q$ is a good
reduction of $E/K$, where $K$ is a number field embedded in
the field of complex numbers, we would expect
the Betti numbers of the space of complex points of $E$ 
to coincide with these values, and indeed, as we
will see in the following section, this is indeed the case.
